\input{pre}
\setbeamercolor{alerted text}{fg=blue}
\setbeamersize{text margin left=4mm,text margin right=1mm} 
\setbeamertemplate{frametitle}[default][center]
\setbeamertemplate{navigation symbols}{
	\usebeamerfont{footline}%
    \usebeamercolor[fg]{footline}%
    \hspace{1em}%
    {{\small презентация доступна: \href{https://bit.ly/2NT0OUN}{\textbf{bit.ly/2NT0OUN}}}
    \hspace{5cm}
    \insertframenumber/\inserttotalframenumber\vspace{0.5mm}}}
\title[]{Акустическая фонетика за 90 минут}
\author[]{Г. Мороз}
\date{30 июля, 2018}
\begin{document}
\frame{\titlepage}
\section{мифы}
\begin{frame}{Список мифов, чтобы мы были на одной волне:}
\begin{itemize}
\item люди отличаются от животных своей способностью мыслить и коммуницировать (по другой версии --- шутить и смеяться) \pause
\begin{itemize}
\item животные много общаются и передают друг другу информацию \pause
\item животные способны мыслить \pause
\item животные на каком-то уровне могут выучить человеческие языки \pause
\end{itemize}
\item люди общаются при помощи слов и речи \pause
\begin{itemize}
\item контекст, мимика, жесты говорят порой больше, чем слова \pause
\item люди, слава богу, научились писать \pause
\item бывают жестовые языки
\end{itemize}
\item хорошее распознавание лучше сделает программист, а не лингвист\pause . К сожалению, это не миф...
\end{itemize}
\end{frame}

\section{}
\begin{frame}
{\huge Спасибо, что дослушали!\bigskip\\
agricolamz@gmail.com
\vspace{-130pt}}
\end{frame}
\end{document}
